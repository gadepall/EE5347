\documentclass[journal,12pt,onecolumn]{IEEEtran}
\usepackage{amsmath}
\usepackage{amssymb}
\usepackage{enumerate}
\usepackage[utf8]{inputenc}
\usepackage{multicol}
\usepackage{gensymb}
\usepackage{nopageno}
\usepackage{mathtools}
\usepackage{siunitx}
%\usepackage{graphicx}
%DeclareUnicodeCharacter{2212}{+}
\begin{document}
\centering \textbf{EE608  Adaptive Signal Processing}\\
\medskip
\centering{Problem Set 5}\\
\bigskip
\begin{enumerate}
\item \textit{Complex LMS Algorithm}\\
Let all the variables in the algorithm be complex. Now, wherever  you have transpose terms like $A^T$, it will get replaced by $A^H={(A^*)}^T$. Let\\
\begin{align*}
&\text{Tap input vector be X(n)}={X_R}(n)+j{X_I}(n)\\
&\text{Desired Response be d(n)}={d_R}(n)+jd_I(n)\\
&\text{Tap Weight Vector be W(n)}={W_R}(n)+j{W_I}(n)\\
&\text{Estimation Error be e(n)}=e_R(n)+j{e_I}(n)\\    
\end{align*}
\medskip
Show that
\begin{align*}
W_R(n+1)& ={W_R}(n)+\mu[e_R(n)X_R(n)-e_I(n)X_I(n)]\\
W_I(n+1)& =W_I(n)+\mu[e_R(n)X_I(n)+e_I(n)X_R(n)]\\
\end{align*}
where
\begin{align*}
e_R(n)& =d_R(n)-W_{R}^{T}{X_R(n)}+W_{I}^{T}(n)X_I(n)\\
e_I(n)& =d_I(n)-W_{R}^{T}{X_I(n)}-W_{I}^{T}(n){X_R(n)}\\
\end{align*}
\smallskip
Note: You will need to use derivative of complex quantities w.r.t to real and imaginary parts.\\
Let $f(W)$ be a scalar complex function of a complex column vector $W=W_R+jW{_I},$and $W_R=[w_{R}^{1}...w_{R}^{M}]^T,W_I=[w_{I}^{1}...w_{I}^{M}]^T.$ Now the derivative of $f(W)$ w.r.t W is defined as:
\bigskip
\begin{align*}
\frac{\partial{f(W)}}{\partial{W}}=\begin{bmatrix}{\frac{\partial{f(W)}}{\partial{w}_R^1}}-{j\frac{\partial{f(W)}}{\partial{w}_I^1}} \\[12pt]
{\frac{\partial{f(W)}}{\partial{w}_R^2}}-{j\frac{\partial{f(W)}}{\partial{w}_I^2}} \\
\vdots \\
{\frac{\partial{f(W)}}{\partial{w}_R^M}}-{j\frac{\partial{f(W)}}{{w}_I^M}}
\end{bmatrix}
\end{align*}
and the conjugate derivative is defined as\\
\begin{align*}
\frac{\partial{f(W)}}{\partial{W}^{*}}=\begin{bmatrix}{\frac{\partial{f(W)}}{\partial{w}_R^1}}+{j\frac{\partial{f(W)}}{\partial{w}_I^1}} \\[12pt]
{\frac{\partial{f(W)}}{\partial{w}_R^2}}+{j\frac{\partial{f(W)}}{\partial{w}_I^2}} \\
\vdots \\
{\frac{\partial{f(W)}}{\partial{w}_R^M}}+{j\frac{\partial{f(W)}}{{w}_I^M}}
\end{bmatrix}
\end{align*}
\bigskip
\item \textbf{Simulation:}This problem deals with \textit{adaptive noise cancellation} using the LMS algorithm. You will need two data streams: one corresponding to signal plus noise and the other corresponding to noise. You could synthesize such
data but this would be too far from the real situation. Thus
\medskip
\begin{enumerate}[(a)]
\item Use the speech data files provided on the course web page.\\
\smallskip
The files are for speech plus noise, and only noise. You will notice that you cannot decipher the speech when you play the speech plus noise file; but upon proper implementation of the noise cancellation algorithm you will be able to decipher the speech.
\begin{enumerate}[i.]
\medskip
\item noise3.wav and signal noise3.wav form one par
\item snp.wav and n.wav from one pair
\item look for other such pairs on the web; these can also be included in the noise cancellation simulation data
set.
\end{enumerate}
\bigskip
\item OR (optional) your group should capture its own data streams.\\
You need to generate one stream as: speech data plus noise, and the other stream as only noise. You could
use two microphones to capture the data streams. The main catch is in capturing both the data streams simultaneously.\\
Have at least two set of data; each with a different SNR (one with low SNR and the other with high SNR).
\end{enumerate}
\end{enumerate}
\end{document}
