\documentclass[journal,12pt,onecolumn]{IEEEtran}
\usepackage{amsmath}
\usepackage{amssymb}
\usepackage{enumerate}
\usepackage[utf8]{inputenc}
\usepackage{multicol}
\usepackage{gensymb}
\usepackage{nopageno}
\usepackage{mathtools}
\usepackage{siunitx}
\begin{document}
\centering \textbf{EE608  Adaptive Signal Processing}\\
\medskip
\centering{Problem Set 8}\\
\medskip
\begin{enumerate}
\item Implement Sato’s blind channel equalizer for:
\medskip
\begin{enumerate}[(a)]
\item BPSK data symbols and channel as specified\\
\bigskip
$h(n)=\begin{cases}
\bigskip
\frac{1}{2}[1+cos(\frac{2\pi}{F}{(n-2)})]& n=1,2,3\\
0 & otherwise
\end{cases}$\\
with $F=4.5$ or greater.
\item QPSK data, with the channel as above.
\item BPSK data, but with a narrow band channel
$h(n)=\begin{cases}
\bigskip
{1}&  n=0,1,... 9\\
0&  otherwise
\end{cases}$\\
\medskip
\item QPSK data with narrow band channel.
\item A channel of your choice.\\
In all cases specify the variance of additive noise.
\medskip
Compare the results of the Sato’s blind algorithm with the previous simulation where one had used the knowledge
of the channel.\\
\end{enumerate}
\medskip
\item Implement the \textit{Transform Based} LMS algorithm in the one step predictive mode. Use the data set for the LMS
algorithm from Home Work No.3, Problem 8. Use any real transform; DCT or DWT. Comment on the convergence
and MSE performance of the transform based LMS algorithm.






























\end{enumerate}
\end{document}